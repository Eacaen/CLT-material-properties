\documentclass{article}
\usepackage[UTF8]{ctex}
\usepackage{hyperref}
\usepackage{mathtools,amsmath,amssymb}
\usepackage{graphicx}
\usepackage{listings} 
\usepackage{tocloft}
\hypersetup{colorlinks=true,linkcolor=black}
\begin{document}
\title{ Laminate\_Analysis  Introduction \& Handbook }
\author{Eacaen \\ tianyunhu@gmail.com }
\maketitle
\includegraphics[width = .9\textwidth]{space.jpg}
\newpage
\tableofcontents                        
\newpage
\section{Composite Material Calculation with CLT}
	\subsection{class Fibre}
		The class designed to define and change the fibre material's properties. 
		\subsubsection{eg.}
		\begin{lstlisting}[language={Python}] 
	f = Fibre(Ef1=74000,Ef2=74000,Gf12=30800,\
		vf21=0.2,density=2.55)
		\end{lstlisting}

	\subsection{class Matrix}
		The class designed to define and change the matrix material's properties. 

		\subsubsection{eg.}	
		\begin{lstlisting}[language={Python}] 
	m = Matrix(Em=300,Gm=1222,vm=0.35,density=1.18)
		\end{lstlisting}

	\subsection{class Lamina}
		The class is used to define the signal lamina, its properties and engineering constants.

		In the lamina initialization, you need to give the Elastic moduli $E_1,E_2$, the shear moduli $G_{12}$, the major Poisson's ration $v_{21}$ or $v_{12}$. And if you want to do the strength failure analysis, you need to give the tensile strength and compressive strength parallel to the fibre $X_t,X_c$, the tensile strength and compressive strength of the unidirectional layer transverse to the fibre $Y_t,Y_c$, and the shear strength $S_{21}$. Don't forget to define the angle and thickness of the lamina.(All the constants' default value are zero 0 )

		And the second way to define from the fibre and matrix materials have been defined above, read the example to get more.

		\subsubsection{eg. define lamina directly}	
		\begin{lstlisting}[language={Python}] 
	a=Lamina(E1=5.4e4,E2=0.001,G12=0.001,v21=0.25,
	Xt=1.05e3,Xc=1.05e3,Yt=28,Yc=140,S=42,
	angle=0,thickness=1)
		\end{lstlisting} 

		\subsubsection{eg. define lamina by fibre and matrix }
		\begin{lstlisting}[language={Python}] 
	a = Lamina(fibre=f , angle = 0 ,thickness=1 )
	b = Lamina(matrix=m, angle = 90,thickness=10.0)
		\end{lstlisting} 

		After define the lamina,you can get the matrix of the lamina,like matrix\_Q or matrix\_Qbar, you can get more if you look for the source code.
		

	\subsection{class Laminate}
		The class is used to define the Laminate.

		After the laminate initialization, you need to add the lamina as the lay up order of the laminate order.Then update the laminate you define,you can get the matrix\_ABD, the thickness and so on.

		\subsubsection{eg.}	
		\begin{lstlisting}[language={Python}] 
	LA = Laminate()  #laminate initialization
	LA.add_Lamina(a) #add lamina
	LA.add_Lamina(a)
	LA.update()      #update and start calculation
	LA.repalce_Lamina(0,b)
	#replace the first(start form 0) lamina_a with the lamina_b
	LA.remove_Lamina(0) #remove the first lamina
	LA.update() #need to update again when change the lamina
		\end{lstlisting} 

	\subsection{class Loading}
		The class is used to define the Loading apply to the laminate.

		After the laminate update, you can apply the load defined to the laminate, and get the stress and strain $\sigma_1$, $\sigma_2$,$\tau_{12}$ in the lamina COS, the stress and strain $\sigma_x$, $\sigma_y$,$\tau_{xy}$ in the laminate COS.

		\subsubsection{eg.}	
		\begin{lstlisting}[language={Python}] 
	Load = Loading(1,0,0,0,0,0) #load define
	Load.apple_to(LA) # load apply to laminate
	Load.change_Load(2,0,0,0,0,0) #change load
	Load.apple_to(LA) # load apply to laminate again
		\end{lstlisting}

	\subsection{class Failure\_Criterion}
		The class is used text the strength of the laminate use Failure\_Criterion.You can choose which lamina to test or whole laminate. 

		\subsubsection{eg.}	
		\begin{lstlisting}[language={Python}] 
	c=Failture_Criterion()  #Criterion initialization
	c.Tsai_Hill(Load,layer_num=1) 
	#choose the strength criterion and which lamina or all
	r= c.ret_list #get the answers
		\end{lstlisting}
		 	
	\subsection{laminate\_Tools Introduction}
		This part can print, save and plot the results in excel file format, you should give a name of the results you want to save.(the default do not save,and the name without `.' in). When calling the function, you can also choose which lamina's results to show or show all the laminate. 

		\subsubsection{eg.}	
		\begin{lstlisting}[language={Python}] 
	print Report_strain(Load,mode='12')
	plot_strain(Load,mode='xy',max_ten=None,mode2='1')
		\end{lstlisting}		





\end{document}